\documentclass{amsart}
\usepackage{sean}

\title{Differential Geometry Website Plan}
\author{Sean Richardson \\\today}

\begin{document}
\maketitle

\section*{Guidelines}
\begin{itemize}
	\item Straightforward and concise.
	\item Modular so that people can use just that page if needed.
	\item Motivate each topic well.
	\item However, keep motivation brief and possibly visual.
	\item Incorporate 
	\item Hide boring computations so that when glancing over, the overall logic is obvious.
	\item Simple and minimal design.
	\item Backbone is always interactive visuals... generally first work on level (1) explanations, build interactive visuals for that, then build higher levels ontop. For rigorous level might sometimes use visuals after learning the topic.
\end{itemize}

\section*{Different Options}
\begin{enumerate}
	\item Intuitive level with visuals. This could often act as the motivation for the topic.
	\item Basic, undergrad-ish level 
	\item Formal level.
\end{enumerate}

\section*{Riemannian Geometry Outline}
\begin{itemize}
	\item Background
	\begin{itemize}
		\item Necessary multivariable calculus
		\item Explanations for lengths, areas, angles, etc. in curvilinear coorinates. Then I can just use formulas without explanation in motivation?
	\end{itemize}
	\item Definition of Riemannian manifold.
	\begin{enumerate}
		\item Motivation is visual with sphere... want to calculate lengths on the sphere, so bestow the map of a globe with geometric information (line element?). Keep brief. Doesn't need to be precise?
		\item[(1.5)] Maybe also give inner product $g$ visual on sphere.
		\item Practical informal ``definition''. Surface and coordinates with line element like Iva does.
		\item Elementary formal definition using coordinate charts.
		\item Formal definition with smooth manifold theory as in Lee (requires extra background).
	\end{enumerate}
	\item The sphere as Riemannian manifold.
	\begin{enumerate}
		\item[(1-2)] Manipulations with line element.
		\item[(2-3)] Computations with inner product.
		\item[(3)]
	\end{enumerate}
	\item Using Riemannian metric to compute lengths, angles, distances, volumes.
	\item Geodesic equation (for 1-2 certainly). (3) might take different path, not sure.
	\item .
\end{itemize}

\section*{Thoughts}
\end{document}